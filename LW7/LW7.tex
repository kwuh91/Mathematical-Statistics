\documentclass[a4paper, 14pt]{extarticle}

\usepackage{../latexDependencies/misc/preamble2}

\geometry{a4paper}

% Название дисциплины
\newcommand{\subject}{Теория вероятности и математическая статистика} 

% Тип работы
% lab - для лабораторной работы 
% hw  - для домашней     работы
\newcommand{\task}{lab} 

% Номер работы
\newcommand{\taskNumber}{7} 

% Название работы
\newcommand{\taskNameOne}{???} 
\newcommand{\taskNameTwo}{???} 

% Имя студента
\newcommand{\studentName}{Очкин Н.В.}

% Имя преподававателя
\newcommand{\teacherName}{Облакова Т.В.}

% Группа
\newcommand{\group}{ФН11-52Б}

% Вариант
\newcommand{\variant}{9}

\begin{document}

\graphicspath{ {../latexDependencies/images} } 
\input{../latexDependencies/frontmatter/titlepage2}

\newgeometry{left=25mm, right=25mm, top=20mm, bottom=20mm}

\graphicspath{ {../latexDependencies/images/LW7} }

% Customize section, subsection, subsubsection and paragraph styles
\titleformat{\section}
  {\normalfont\large\bfseries}{\thesection}{1em}{}

\titleformat{\subsection}
  {\normalfont\normalsize\bfseries}{\thesubsection}{1em}{}

\titleformat{\subsubsection}
  {\normalfont\small\bfseries}{\thesubsubsection}{1em}{}

\titleformat{\paragraph}
  {\small\small\bfseries}{\theparagraph}{1em}{}

\thispagestyle{empty}

\null\newpage

% \setcounter{tocdepth}{5}
% \setcounter{secnumdepth}{5}

% \pagenumbering{roman}

% \tableofcontents
% \newpage

\pagenumbering{arabic}
\setcounter{page}{1}

\setstretch{1}
\linespread{1.1}

\setlength{\parindent}{0pt}

\fontsize{12pt}{16pt}\selectfont

\definecolor{myblue}{HTML}{0A88C2}
\definecolor{myred}{HTML}{FF1B1C}
\definecolor{mygreen}{HTML}{386641}

\lstdefinestyle{mystyle}{
    basicstyle=\ttfamily\footnotesize,
    keywordstyle=\color{myblue},
    stringstyle=\color{myred},
    commentstyle=\color{green!50!black},
    showstringspaces=false,
    frame=leftline, 
    framesep=10pt, 
}

% Set the style for Python code
\lstset{style=mystyle, extendedchars=\true}

% --------------------------------------START--------------------------------------

Для нахождения функции распределения (CDF) из функции плотности вероятности (PDF) равномерного распределения, давайте рассмотрим стандартное равномерное распределение на отрезке \([a, b]\).

PDF равномерного распределения на отрезке \([a, b]\) определяется как:

\[ f(x) = \begin{cases}
\frac{1}{b-a} & \text{если } a \leq x \leq b \\
0 & \text{иначе}
\end{cases} \]

CDF \(F(x)\) определяется как интеграл PDF от \(-\infty\) до \(x\):

\[ F(x) = \int_{-\infty}^{x} f(t) \, dt \]

Рассмотрим три случая для \(x\):

1. **Если \(x < a\)**:
   \[ F(x) = \int_{-\infty}^{x} f(t) \, dt = \int_{-\infty}^{x} 0 \, dt = 0 \]

2. **Если \(a \leq x \leq b\)**:
   \[ F(x) = \int_{-\infty}^{x} f(t) \, dt = \int_{-\infty}^{a} 0 \, dt + \int_{a}^{x} \frac{1}{b-a} \, dt \]
   \[ F(x) = 0 + \frac{1}{b-a} \int_{a}^{x} \, dt = \frac{1}{b-a} [t]_{a}^{x} = \frac{1}{b-a} (x - a) = \frac{x - a}{b - a} \]

3. **Если \(x > b\)**:
   \[ F(x) = \int_{-\infty}^{x} f(t) \, dt = \int_{-\infty}^{a} 0 \, dt + \int_{a}^{b} \frac{1}{b-a} \, dt + \int_{b}^{x} 0 \, dt \]
   \[ F(x) = 0 + \frac{1}{b-a} \int_{a}^{b} \, dt + 0 = \frac{1}{b-a} [t]_{a}^{b} = \frac{1}{b-a} (b - a) = 1 \]

Таким образом, CDF равномерного распределения на отрезке \([a, b]\) будет:

\[ F(x) = \begin{cases}
0 & \text{если } x < a \\
\frac{x - a}{b - a} & \text{если } a \leq x \leq b \\
1 & \text{если } x > b
\end{cases} \]

Это и есть функция распределения (CDF) для равномерного распределения на отрезке \([a, b]\).

\end{document}

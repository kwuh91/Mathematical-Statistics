\documentclass[a4paper, 14pt]{extarticle}

\usepackage{../latexDependencies/misc/preamble2}

\geometry{a4paper}

% Название дисциплины
\newcommand{\subject}{Теория вероятности и математическая статистика} 

% Тип работы
% lab - для лабораторной работы 
% hw  - для домашней     работы
\newcommand{\task}{lab} 

% Номер работы
\newcommand{\taskNumber}{7} 

% Название работы
\newcommand{\taskNameOne}{Критерий согласия для проверки простой} 
\newcommand{\taskNameTwo}{непараметрической гипотезы} 

% Имя студента
\newcommand{\studentName}{Очкин Н.В.}

% Имя преподававателя
\newcommand{\teacherName}{Облакова Т.В.}

% Группа
\newcommand{\group}{ФН11-52Б}

% Вариант
\newcommand{\variant}{9}

\begin{document}

\graphicspath{ {../latexDependencies/images} } 
\input{../latexDependencies/frontmatter/titlepage2}

\newgeometry{left=25mm, right=25mm, top=20mm, bottom=20mm}

\graphicspath{ {../latexDependencies/images/LW7} }

% Customize section, subsection, subsubsection and paragraph styles
\titleformat{\section}
  {\normalfont\large\bfseries}{\thesection}{1em}{}

\titleformat{\subsection}
  {\normalfont\normalsize\bfseries}{\thesubsection}{1em}{}

\titleformat{\subsubsection}
  {\normalfont\small\bfseries}{\thesubsubsection}{1em}{}

\titleformat{\paragraph}
  {\small\small\bfseries}{\theparagraph}{1em}{}

\thispagestyle{empty}

\null\newpage

% \setcounter{tocdepth}{5}
% \setcounter{secnumdepth}{5}

% \pagenumbering{roman}

% \tableofcontents
% \newpage

\pagenumbering{arabic}
\setcounter{page}{1}

\setstretch{1}
\linespread{1.1}

\setlength{\parindent}{0pt}

\fontsize{12pt}{16pt}\selectfont

\definecolor{myblue}{HTML}{0A88C2}
\definecolor{myred}{HTML}{FF1B1C}
\definecolor{mygreen}{HTML}{386641}

\lstdefinestyle{mystyle}{
    basicstyle=\ttfamily\footnotesize,
    keywordstyle=\color{myblue},
    stringstyle=\color{myred},
    commentstyle=\color{green!50!black},
    showstringspaces=false,
    frame=leftline, 
    framesep=10pt, 
}

% Set the style for Python code
\lstset{style=mystyle, extendedchars=\true}

% --------------------------------------START--------------------------------------

\section*{Задание}\vspace{-20pt}\rule{\linewidth}{0.1mm}

Постройте с помощью стохастического эксперимента на основе указанной метрики приближенный 
критерий для проверки основной гипотезы. 
Найдите критические значения $D_{\text{кр}}$ для трех уровней значимости 
$\alpha = 0.1, 0.05 \text{ и } 0.01$.

Протестируйте критерий на трех-четырех примерах и сформулируйте выводы. 

\section*{Исходные данные}\vspace{-20pt}\rule{\linewidth}{0.1mm}

\end{document}

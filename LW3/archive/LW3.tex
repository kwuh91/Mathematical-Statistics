\documentclass[a4paper, 14pt]{extarticle}

\usepackage{../latexDependencies/misc/preamble2}

\geometry{a4paper}

% Название дисциплины
\newcommand{\subject}{Теория вероятности и математическая статистика} 

% Тип работы
% lab - для лабораторной работы 
% hw  - для домашней     работы
\newcommand{\task}{lab} 

% Номер работы
\newcommand{\taskNumber}{3} 

% Название работы
\newcommand{\taskNameOne}{Моделирование выборки из абсолютно непрерывного } 
\newcommand{\taskNameTwo}{закона распределения методом обратных функций.} 

% Имя студента
\newcommand{\studentName}{Очкин Н.В.}

% Имя преподававателя
\newcommand{\teacherName}{Облакова Т.В.}

% Группа
\newcommand{\group}{ФН11-52Б}

% Вариант
\newcommand{\variant}{9}

\begin{document}

\graphicspath{ {../latexDependencies/images} } 
\input{../latexDependencies/frontmatter/titlepage2}

\newgeometry{left=25mm, right=25mm, top=20mm, bottom=20mm}

\graphicspath{ {../latexDependencies/images/LW2} }

% Customize section, subsection, subsubsection and paragraph styles
\titleformat{\section}
  {\normalfont\large\bfseries}{\thesection}{1em}{}

\titleformat{\subsection}
  {\normalfont\normalsize\bfseries}{\thesubsection}{1em}{}

\titleformat{\subsubsection}
  {\normalfont\small\bfseries}{\thesubsubsection}{1em}{}

\titleformat{\paragraph}
  {\small\small\bfseries}{\theparagraph}{1em}{}

\thispagestyle{empty}

\null\newpage

\setcounter{tocdepth}{5}
\setcounter{secnumdepth}{5}

\pagenumbering{roman}

\tableofcontents
\newpage

\pagenumbering{arabic}
\setcounter{page}{1}

\setstretch{1}
\linespread{1.1}

\setlength{\parindent}{0pt}

\fontsize{12pt}{16pt}\selectfont

% --------------------------------------START--------------------------------------

\section{Задание}\vspace{-20pt}\rule{\linewidth}{0.1mm}

\begin{enumerate}
  \item Для данного $n$ методом обратных функций смоделируйте выборку 
  из закона распределения с заданной плотностью  $p(x)$.
  \item Для полученной выборки найдите гистограмму относительных частот. 
  Постройте на одном рисунке графики теоретической плотности $p(x)$ и 
  гистограмму относительных частот.
  \item Вычислите выборочное среднее и выборочную дисперсию и сравните с 
  истинными значениями этих характеристик.
  \item Используя неравенство \high{Dvoretzky-Kiefer-Wolfowitz}, 
  постройте 90\% доверительный интервал для функции распределения $F(x)$.
\end{enumerate}

Приведите графическую иллюстрацию

\section{Исходные данные}\vspace{-20pt}\rule{\linewidth}{0.1mm}

\begin{equation*}
  \text{Вариант: }9 \qquad n: 120
\end{equation*}
\begin{equation}
  \scalebox{1.25}{$p(x) = \cfrac{1}{\sqrt{0.4 \pi}x} e^{-(\ln{x} - 2)^2 / 0.4}, \quad x > 0$}
\end{equation}

\section{Решение}

\subsection{Часть 1}\vspace{-20pt}\rule{\linewidth}{0.1mm}
Для данного $n$ методом обратных функций смоделируйте выборку 
из закона распределения с заданной плотностью  $p(x)$.\\

\subsubsection{Функция распределения}\vspace{-20pt}\rule{\linewidth}{0.1mm}

Найдем функцию распределения
\begin{equation}
  F_X(x) = \int_{-\infty}^{x} f_X(t)dt, \quad \text{где}
\end{equation}
$f_X$ - плотность распределения.\\

\paragraph{Метод интегрирования Монте-Карло}\vspace{-20pt}\rule{\linewidth}{0.1mm}

Для вычисления интеграла воспользуемся численным методом интегрирования Монте-Карло
\begin{equation}
  \int_{a}^{b} f(x) dx \approx \cfrac{b-a}{N} \sum_{i=1}^{N} f(u_i), \quad \text{где}
\end{equation}
$u$ - равномерно распредленная на отрезке интегрирования $[a, b]$ случайная величина.\\

Геометрическая интерпретация данного метода похожа на известный детерминистический метод, 
с той разницей, что вместо равномерного разделения области интегрирования на маленькие интервалы 
и суммирования площадей получившихся «столбиков» мы забрасываем область интегрирования случайными 
точками, на каждой из которых строим такой же «столбик», определяя его ширину как 
$\cfrac{b - a}{N}$, и суммируем их площади.\\

Точность оценки данного метода зависит только от количества точек $N$.\\

\paragraph{Линейный конгруэнтный метод}\vspace{-20pt}\rule{\linewidth}{0.1mm}

Для генерации случайных величин воспользуемся одним из методов генерации псевдослучайных чисел - 
\textbf{Линейным конгруэнтным методом}.\\
Суть метода заключается в вычислении последовательности случайных чисел $X_n$, полагая
\begin{equation}
  X_{n+1} = (aX_n + c)\hspace{3pt} \text{mod} \hspace{3pt} m, \quad \text{где}
\end{equation}
$m$ - модуль ($m \geq 2$); \\
$a$ - множитель ($0 \leq a < m$); \\
$c$ - приращение ($0 \leq c < m$); \\
$X_0$ - начальное значение ($0 \leq X_0 < m$).\\

За значениями параметров обратимся к [\ref{item:source1}].
\begin{equation}
  m = 2^{(60)} - 93 \qquad a = 561860773102413563 \qquad c = 0.
\end{equation}
В случае когда $c = 0$, метод называют \textbf{мультипликативным конгруэнтным методом}.

\subsubsection{Реализация численного нахождения функции распределения}

\paragraph{Реализация ЛКМ}\vspace{-20pt}\rule{\linewidth}{0.1mm}

Реализуем на языке программирования \high{python} линейный конгруэнтный метод (4), используя 
параметры (5):\\

\begin{lstlisting}[language=Python, caption={Реализация ЛКМ}, label={lst:LCG}]
class LCG:
  def __init__(self, 
               seed, a=561860773102413563, c=0, m=2**60-93):
      self.seed = seed
      self.a = a
      self.c = c
      self.m = m
      self.state = seed

  def next(self):
      self.state = (self.a * self.state + self.c) % self.m
      return self.state / self.m # Normalize to [0, 1)
  
  def next_in_range(self, a, b):
      return a + (b - a) * self.next()
  \end{lstlisting}

\paragraph{Реализация метода Монте-Карло}\vspace{-20pt}\rule{\linewidth}{0.1mm}

Теперь реализуем интегрирование методом Монте-Карло, используя ранее 
описаннный ЛКМ (листинг \ref{lst:LCG}):\\

\begin{lstlisting}[language=Python, caption={Реализация метода Монте-Карло}, label={lst:MonteCarlo}]
class MonteCarlo:
  def __init__(self, N, PRNG_object):
      self.N = N
      self.PRNG = PRNG_object
  
  def integrate(self, f, a, b):
      mult = (b - a) / self.N
      
      generatedValues = []
      for _ in range(self.N):
          randomArg = self.PRNG.next_in_range(a, b)
          randomFuncVal = f(randomArg)

          generatedValues.append(randomFuncVal)
      
      return mult * sum(generatedValues)
\end{lstlisting}

\paragraph{Реализация функции распределения}\vspace{-20pt}\rule{\linewidth}{0.1mm}

Объединим теперь (2) и (3) и получим:
\begin{equation}
  F_X(x) = \int_{-\infty}^{x} f_X(t)dt \approx \cfrac{x + \infty}{N} 
  \sum_{i=1}^N f(u_i), \quad \text{где}
\end{equation}
$u_i$ ищем в соответствии с (4).\\

В общем случае пришлось бы производить замену, чтобы свести бесконечные пределы в конечные, 
однако в нашем случае это не требуется, тк (1) определена при $x > 0$.\\
Подставляя (1) в (6) и (5) в (4):
\begin{equation}
  F_X(x) = \int_{0}^{t} \cfrac{1}{\sqrt{0.4 \pi}t} e^{-(\ln{t} - 2)^2 / 0.4} dx 
  \approx \cfrac{x}{N} \sum_{i=1}^{N} \cfrac{1}{\sqrt{0.4 \pi}u_i} e^{-(\ln{u_i} - 2)^2 / 0.4}, 
  \quad \text{где}
\end{equation} 
$u_{i} = (561860773102413563 \cdot u_{i-1})\hspace{3pt} \text{mod} 
\hspace{3pt} 2^{60} - 93$ \\\\

При программной реализации нас не сильно интересуют конкретные начальное значение 
в ЛКМ и значение $N$ в методе Монте-Карло. \\
Первое выбирается так, чтобы $x_0 \neq 0$. Это необходимо для того, чтобы 
последовательность была полной длины, т.е. имела максимальную периодичность 
при генерации чисел. Обычно используют случайное или произвольно выбранное 
значение из множества $\{1, ..., m - 1\}$ [\ref{item:source1}]. \\
Второе, как уже было сказано ранее, отвечает за точность полученной оценки метода, так 
что чем оно больше, тем лучше.

\begin{lstlisting}[language=Python, caption={Реализация функции распределения}, label={lst:cdf}]
if __name__ == '__main__':
  lcg        = LCG(seed=42)
  monteCarlo = MonteCarlo(100000000, lcg)
  
  def pdf(x): # f_X
      return 1 / (math.sqrt(0.4 * math.pi) * x) \
          * math.exp(-(math.log(x) - 2)**2 / 0.4)

  def cdf(x): # F_X
      return monteCarlo.integrate(pdf, 0, x)
\end{lstlisting}
\vspace{10pt}
где классы \textbf{LCG} и \textbf{MonteCarlo} представлены в листингах 
\ref{lst:LCG} и \ref{lst:MonteCarlo} соответственно.
 
\subsubsection{Обратная функция}

\paragraph{Метод Ньютона}\vspace{-20pt}\rule{\linewidth}{0.1mm}

Для нахождения обратной функции воспользуемся методом касательных (Ньютона). \\
Рабочая формула
\begin{equation*}
  x_{n+1} = x_n - \cfrac{f(x_n)}{f'(x_n)}
\end{equation*}
Вообще говоря, метод используется для нахождения корня заданной функции. Так что 
для нахождения обратной функции $y = f(x)$, т.е. $x = f^{-1}(y)$ будем искать 
решение уравнения: $f(x) - y = 0$
\begin{equation}
  x_{n+1} = x_n - \cfrac{f(x_n) - y}{(f(x_n) - y)'_x} = 
  x_n - \cfrac{f(x_n) - y}{f'(x_n)}
\end{equation}
Погрешность $\varepsilon$ возьмем равной 0.001.

\paragraph{Метод центральных разностей}\vspace{-20pt}\rule{\linewidth}{0.1mm}

Производные будем искать методом центральных разностей.\\
Рабочая формула
\begin{equation}
  f'(x) \approx \cfrac{f(x+h) - f(x-h)}{2h}
\end{equation}
Погрешность определяется как $O(h)$, $h$ примем равной 1e-6.\\

Подставив (9) в (8), получим:
\begin{equation}
  x_{n+1} = x_n - \cfrac{(f(x_n) - y) \cdot 2 h}{f(x_n + h) - f(x_n - h)}
\end{equation}

\subsubsection{Реализация численного обратной функции}

\paragraph{Реализация ЛКМ}\vspace{-20pt}\rule{\linewidth}{0.1mm}

\newpage
\section{Список использованных источников}\vspace{-20pt}\rule{\linewidth}{0.1mm}
\begin{enumerate}
  \item \label{item:source1} L'Ecuyer, Pierre (January 1999). "Tables of Linear Congruential Generators of Different Sizes and Good Lattice Structure" - С. 256
\end{enumerate}

\end{document}

% Прежде чем реализовывать вычисление самой функции, заметим, что в пределах 
% интегрирования (2) присутствует бесконечность, что затрудняет интегрирование 
% методом Монте-Карло (3). \\
% Воспользуемся заменой, чтобы свести бесконечные пределы в конечные:
% \begin{gather*}
%   F_X(x) = \int_{-\infty}^{x} f_X(t)dt = \\[1em]
%   = \left[\hspace{5pt}
%     \begin{gathered}
%     t       = \tan(v)                            \\
%     v       = \arctan(t)                         \\
%     dt      = \cfrac{1}{\cos^2(v)} dv            \\
%     x       : v=\arctan{x}                       \\ 
%     -\infty : v=\arctan(-\infty)=-\cfrac{\pi}{2} \\
%     \end{gathered}\,
%   \hspace{5pt}\right] = \\[1em]
%   = \int_{-\cfrac{\pi}{2}}^{\arctan(x)} f_X(\tan(v)) \cdot \cfrac{1}{\cos^2(v)} dv
% \end{gather*}

% Итого получим:
% \begin{equation*}
%   F_X(x) = \int_{-\cfrac{\pi}{2}}^{\arctan(x)} f_X(\tan(v)) \cdot \cfrac{1}{\cos^2(v)} dv
% \end{equation*}

% Обозначим $f_X(\tan(v)) \cdot \cfrac{1}{\cos^2(v)}$ за $g(v)$ и подставим в (3):
% \begin{equation}
%   F_X(x) = \int_{-\cfrac{\pi}{2}}^{\arctan(x)} g(v) dv \approx 
%   \cfrac{\arctan(x) + \frac{\pi}{2}}{N} \sum_{i=1}^{N} g(u_i), \quad \text{где}
% \end{equation}
% $u_i$ ищем в соответствии с (листинг \ref{lst:LCG})

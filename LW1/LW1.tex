\documentclass[a4paper, 14pt]{extarticle}

\usepackage{../latexDependencies/misc/preamble}

\geometry{a4paper}

% Название дисциплины
\newcommand{\subject}{Теория вероятности и математическая статистика} 

% Тип работы
% lab - для лабораторной работы 
% hw  - для домашней     работы
\newcommand{\task}{lab} 

% Номер работы
\newcommand{\taskNumber}{1} 

% Название работы
\newcommand{\taskName}{Первоначальная обработка статистических данных} 

% Имя студента
\newcommand{\studentName}{Очкин Н.В.}

% Имя преподававателя
\newcommand{\teacherName}{Облакова Т.В.}

% Группа
\newcommand{\group}{ФН11-52Б}

% Вариант
\newcommand{\variant}{9}

\begin{document}

\graphicspath{ {../latexDependencies/images} } 
\input{../latexDependencies/frontmatter/titlepage}

% Reset the font size and page margins
\newgeometry{left=25mm, right=25mm, top=20mm, bottom=20mm}
\fontsize{14pt}{16pt}\selectfont

\graphicspath{ {../latexDependencies/images/LW1} }

% --------------------------------------START--------------------------------------

\linespread{1.5}

\section*{Задание}

По данной выборе

\begin{enumerate}
    \item Найдите крайние члены вариационного ряда и размах выборки
    \item Осуществите группировку данных (количество интервалов 
    находим по правилу Стерджеса) 
    \item По сгруппированным данным постройте гистограмму 
    относительных частот
    \item Вычислите выборочное среднее и выборочную дисперсию.
    \item По виду гистограммы определите возможный закон распределения, 
    оцените параметры этого закона по методу моментов, постройте совмещенные 
    графики гистограммы и плотности предполагаемого закона
    \item Найдите эмпирическую функцию распределения и постройте совмещенные 
    графики эмпирической и теоретической функций распределения
\end{enumerate}


\end{document}
